\mychapter{Resultados parciais}
\label{cap:resultados}

% Dados para as simulações da qualificação
% Tempo de simulação de 105 segundos (totalizando 00:01:45)
% Valores utilizados => Observar folha


% Montar tabela com os valores dos parametros alterados para o treinamento e
% para a validação das falhas. Exemplo:
% 
% FAVK => km_treinamento = (0.7 a 1.1)*km_original (PRBS)
%         km_validacao = (0.75 a 1.05)*km_original (PRBS) => nao precisa
%                                                            especificar o valor
%                                                            exato, mas dizer
%                                                            que estava dentro
%                                                            da faixa do
%                                                            treinamento
%         km_resultado = (0.75)*km_original (Fixo)
%
% FSeDG => ganho_treinamento = (1 +- 0.2)*ganho_original (PRBS)
%          ganho_validacao = (1 +- 0.15)*ganho_original (PRBS)
%          ganho_resultado = 0.8*ganho_original (fixo)
%


% Mostrar que foram feitos 3 conjuntos de validacao e depois foram escolhidas as
% melhores redes => Mostrar tabelas com Falso positivo e falso negativo


% Quando for falar sobre o número de regressores escolhidos => Dizer que
% considere que os regressores estão representados internamente ao "sistema de
% identificação" e ao "sistema de DDF" na figura da composição do sistema (final
% do cap:sistema)

\begin{comment}
Neste capítulo será realizada uma análise comparativa dos resultados obtidos nas
duas propostas do sistema de DDF. Ao final, do capítulo a melhor estrutura será
escolhida para que seja feita uma análise mais detalhada dos resultados.
\end{comment}

Neste capítulo serão analisados os resultados parciais obtidos a partir da
implementação da segunda proposta do sistema de DDF. Para isso, em um primeiro
momento será mostrado como se deu a coleta dos dados de treinamento e validação
das redes especialistas e em seguida será feita uma comparação de desempenho. Ao
final do capítulo as melhores estruturas serão selecionadas para que se realize
uma análise um pouco mais detalhada acerca da detecção das falhas.

% ------------------------------------------------------------------------------
\section{Coleta dos dados}

% ------------------------------------------------------------------------------
\subsection{Simulações computacionais}

% ------------------------------------------------------------------------------
%\subsection{Método de Runge-Kutta de 4\textordfeminine ordem}

% ------------------------------------------------------------------------------
\section{Análise das RNAs}

% ------------------------------------------------------------------------------
\subsection{Identificação}

% ------------------------------------------------------------------------------
\subsection{Detecção de falhas}

% ------------------------------------------------------------------------------
\section{Melhores redes}

% ------------------------------------------------------------------------------
\section{Composição final}

% ------------------------------------------------------------------------------
\section{Detecções}

% ------------------------------------------------------------------------------
\section{Comparação das propostas}

% Falhas nos sensores ..........................................................
\begin{figure}[htb]
\footnotesize
\centering
\input{scripts/fsedg}
\vspace{1cm}
\caption{FSeDG (80\%)}
\label{fig:fsedg}
\end{figure}

\begin{figure}[htb]
\footnotesize
\centering
\input{scripts/fsedo}
\vspace{1cm}
\caption{FSeDO (-2 cm)}
\label{fig:fsedo}
\end{figure}

\begin{figure}[htb]
\footnotesize
\centering
\input{scripts/fsesr}
\vspace{1cm}
\caption{FSeSR ($\pm 2\%$)}
\label{fig:fsesr}
\end{figure}

\begin{figure}[htb]
\footnotesize
\centering
\input{scripts/fseq}
\vspace{1cm}
\caption{FSeQ (Ganho = 0)}
\label{fig:fseq}
\end{figure}

% Falhas nos atuadores .........................................................
\begin{figure}[htb]
\footnotesize
\centering
\input{scripts/fadg}
\vspace{1cm}
\caption{FADG (80\%)}
\label{fig:fadg}
\end{figure}

\begin{figure}[htb]
\footnotesize
\centering
\input{scripts/fado}
\vspace{1cm}
\caption{FADO (-0,5 Volts)}
\label{fig:fado}
\end{figure}

\begin{figure}[htb]
\footnotesize
\centering
\input{scripts/fasr}
\vspace{1cm}
\caption{FASR ($\pm 2\%$)}
\label{fig:fasr}
\end{figure}

\begin{figure}[htb]
\footnotesize
\centering
\input{scripts/favk}
\vspace{1cm}
\caption{FAVK (75\%)}
\label{fig:favk}
\end{figure}

\begin{figure}[htb]
\footnotesize
\centering
\input{scripts/faq}
\vspace{1cm}
\caption{FAQ (Ganho = 0)}
\label{fig:faq}
\end{figure}

% Falhas no sistema ............................................................
\begin{figure}[htb]
\footnotesize
\centering
\input{scripts/fsivzt}
\vspace{1cm}
\caption{FSiVzT ($a_{\tiny VZ} = \frac{a_{\tiny MED}}{2}$)}
\label{fig:fsivzt}
\end{figure}

\begin{figure}[htb]
\footnotesize
\centering
\input{scripts/fsivros}
\vspace{1cm}
\caption{FSiVrOS ($a' = \frac{a_{\tiny MED}}{2}$)}
\label{fig:fsivros}
\end{figure}

\begin{figure}[htb]
\footnotesize
\centering
\input{scripts/fsivrgmp}
\vspace{1cm}
\caption{FSiVrGMP (90\%)}
\label{fig:fsivrgmp}
\end{figure}

\begin{figure}[htb]
\footnotesize
\centering
\input{scripts/fsieos}
\vspace{1cm}
\caption{FSiEOS (25\%)}
\label{fig:fsieos}
\end{figure}
