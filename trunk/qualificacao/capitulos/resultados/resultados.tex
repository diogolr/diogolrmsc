\mychapter{Resultados}
\label{cap:resultados}

% Dados para as simulações da qualificação
% Tempo de simulação de 105 segundos (totalizando 00:01:45)
% Valores utilizados => Observar folha

\begin{figure}[htb]
\small
\centering
\subfigure
{
\scalebox{0.57}{\input{scripts/fsedg}}
}
\subfigure
{
\scalebox{0.57}{\input{scripts/fsedg}}
}
\caption{FSeDG (80\%)}
\label{fig:fsedg}
\end{figure}


\begin{tikzpicture}
\scriptsize
\begin{axis}
\addplot+[const plot] coordinates
{(0,0.1)    (0.1,0.15)  (0.2,0.5)   (0.3,0.62)
 (0.4,0.56) (0.5,0.58)  (0.6,0.65)  (0.7,0.6)
 (0.8,0.58) (0.9,0.55)  (1,0.52)};
\end{axis}
\end{tikzpicture}

\begin{pgfpicture}
  \pgfsetlinewidth{5mm}
  \color{red}
  \pgfpathcircle{\pgfpoint{0cm}{0cm}}{10mm} \pgfusepath{stroke}
  \color{black}
  \pgfsetstrokeopacity{0.5}
  \pgfpathcircle{\pgfpoint{1cm}{0cm}}{10mm} \pgfusepath{stroke}
\end{pgfpicture}

\begin{comment}
\begin{figure}
\centering
\input{../dados/deteccao/FSeDG/O4/N28/O4N28T2_qualif.tikz}
\caption{lalala}
\end{figure}
\end{comment}

\begin{figure}[htb]
\centering
\footnotesize
\input{scripts/teste}
\caption{Lalala teste $\alpha$}
\label{fig:teste}
\end{figure}

\begin{figure}[htb]
\centering
\scalebox{0.6}{\input{scripts/teste2}}
\scalebox{0.6}{\input{scripts/teste2}}
\caption{Lalala teste 2}
\label{fig:teste2}
\end{figure}

Este é um texto bem legal \ldots

% ------------------------------------------------------------------------------
\section{Coleta dos dados}

% ------------------------------------------------------------------------------
\subsection{Simulações computacionais}

% ------------------------------------------------------------------------------
%\subsection{Método de Runge-Kutta de 4\textordfeminine ordem}

% ------------------------------------------------------------------------------
\section{Análise das RNAs}

% ------------------------------------------------------------------------------
\subsection{Identificação}

% ------------------------------------------------------------------------------
\subsection{Detecção de falhas}

% ------------------------------------------------------------------------------
\section{Melhores redes}

% ------------------------------------------------------------------------------
\section{Composição final}

% ------------------------------------------------------------------------------
\section{Detecções}

% ------------------------------------------------------------------------------
\section{Comparação das propostas}
