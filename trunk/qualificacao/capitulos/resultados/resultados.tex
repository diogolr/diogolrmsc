\mychapter{Resultados}
\label{cap:resultados}

% Dados para as simulações da qualificação
% Tempo de simulação de 105 segundos (totalizando 00:01:45)
% Valores utilizados => Observar folha

\begin{figure}[htb]
\footnotesize
\centering
\input{scripts/fsedg}
\vspace{1cm}
\caption{FSeDG (80\%)}
\label{fig:fsedg}
\end{figure}


\begin{figure}[htb]
\footnotesize
\centering
\input{scripts/fsedo}
\vspace{1cm}
\caption{FSeDO (-2 cm)}
\label{fig:fsedg}
\end{figure}


\begin{figure}[htb]
\footnotesize
\centering
\input{scripts/fsesr}
\vspace{1cm}
\caption{FSeSR ($\pm 3\%$)}
\label{fig:fsedg}
\end{figure}


\begin{figure}[htb]
\footnotesize
\centering
\input{scripts/fseq}
\vspace{1cm}
\caption{FSeQ}
\label{fig:fsedg}
\end{figure}

% ------------------------------------------------------------------------------
\section{Coleta dos dados}

% ------------------------------------------------------------------------------
\subsection{Simulações computacionais}

% ------------------------------------------------------------------------------
%\subsection{Método de Runge-Kutta de 4\textordfeminine ordem}

% ------------------------------------------------------------------------------
\section{Análise das RNAs}

% ------------------------------------------------------------------------------
\subsection{Identificação}

% ------------------------------------------------------------------------------
\subsection{Detecção de falhas}

% ------------------------------------------------------------------------------
\section{Melhores redes}

% ------------------------------------------------------------------------------
\section{Composição final}

% ------------------------------------------------------------------------------
\section{Detecções}

% ------------------------------------------------------------------------------
\section{Comparação das propostas}
