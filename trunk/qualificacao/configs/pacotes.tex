% Pacotes Principais -----------------------------------------------------------
\usepackage[portuges,brazil]{babel}
\usepackage[utf8]{inputenc}

% Formatação de capítulos ------------------------------------------------------
%\usepackage[Sonny]{fncychap}
%\usepackage{fncychap}
\usepackage{capitulos}

% Figuras e Imagens ------------------------------------------------------------
\usepackage{graphicx}
% Figuras lado a lado
\usepackage{epsfig}
\usepackage{subfigure}

% Tikz
\usepackage{tikz}
\usepackage{pgfplots}
\pgfplotsset{compat=1.3}
\usepackage{tikz-timing}[2009/05/15]

% Utilizar H para inserir as imagens REALMENTE onde eu desejo
\usepackage{float}

% Fontes -----------------------------------------------------------------------
\usepackage[T1]{fontenc}
\usepackage{pslatex}

% Simbolos ---------------------------------------------------------------------
\usepackage{textcomp}

% Tabelas ----------------------------------------------------------------------
%\usepackage{multicol}
\usepackage{multirow}
% Colorir a tabela
\usepackage{colortbl}
% Tabelas com colunas de largura auto ajustável
\usepackage{tabularx}
% Notas de rodapé em tabelas (Pode-se usar o ambiente longtable também -
% Pesquisar exemplo com longtable)
\usepackage{threeparttable}

% Glossário --------------------------------------------------------------------
\usepackage[portuguese,noprefix]{nomencl}
\usepackage{makeglo}

% Outros pacotes ---------------------------------------------------------------
\usepackage{noitemsep}

% Comentários em bloco
\usepackage{verbatim}

% Hiperlinks
\usepackage{hyperref}

% Orientação de página
\usepackage{lscape}

% utilitários matemáticos
\usepackage{amsmath}

% Referências ------------------------------------------------------------------
\usepackage[abbr]{harvard}	% As chamadas são sempre abreviadas
\harvardparenthesis{square}	% Colchetes nas chamadas
\harvardyearparenthesis{round}	% Parêntesis nos anos das referências
\renewcommand{\harvardand}{e}	% Substituir "&" por "e" nas referências
