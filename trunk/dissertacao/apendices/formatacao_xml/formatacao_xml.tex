\chapter{Arquivos XML do \textbf{\textit{Simddef}}}
\label{ap:formatacao_xml}

% ------------------------------------------------------------------------------
\section{Arquivos XML}
XML ({\it eXtensible Markup Language}) é uma linguagem de marcação simples e
altamente flexível, criada pela W3C e derivada da {\it Standard Generalized
Markup Language} (SGML -- ISO 8879). Essa linguagem foi idealizada para gerar
linguagens de marcação para necessidades especiais e sua principal
característica é que as linguagens desconhecidas ou de pouco uso podem ser
facilmente definidas sem que haja necessidade de serem submetidas aos comitês de
padronização.

A formatação dos arquivos XML permite que os dados sejam organizados de forma
hierárquica, possibilitando que sejam criados arquivos para sua validação. Além
disso, o formato não depende de plataformas de {\it hardware} ou de {\it
software}, de tal forma que um banco de dados pode, através de uma aplicação,
escrever ou ler um arquivo desse tipo sem maiores complicações.

Por definição, um documento XML é composto por um conjunto de caracteres que,
quando dispostos de forma adequada, compõem uma árvore organizacional dos dados.
A estrutura gerada possui, dentre outros, conjuntos de {\it tags}, {\it
elementos} e {\it atributos}.

As {\it tags}, nada mais são do que as marcações que iniciam com \verb|<| e
terminam com \verb|>|, podendo ser divididas em {\it start-tags}, como por
exemplo \verb|<secao>|, {\it end-tags}, como por exemplo \verb|</secao>| e {\it
empty-element-tags}, como por exemplo \verb|<secao />|.

Os {\it elementos}, por sua vez, são as marcações dos componentes lógicos do
documento, que iniciam com uma {\it start-tag} e encerram com uma {\it end-tag},
podendo ainda serem compostos apenas por uma {\it empty-element-tag}. Os
caracteres entre as {\it tags}, quando existirem, constituem o conteúdo do
elemento e podem conter outros tipos de marcação, inclusive outros elementos,
dando origem aos {\it elementos filhos}. Um exemplo de elemento seria
\verb|<fala>Olá Mundo</fala>| ou apenas \verb|<quebra_linha />|

Já os {\it atributos} são marcações compostas de um par {\it nome}/{\it
valor}, que existem no contexto das {\it start-tags} ou das {\it
empty-element-tags}. Como exemplo, no trecho 
\verb|<imagem end="foto.jpg" largura="100px" />|, existem dois atributos,
denominados \verb|end|, cujo valor é \verb|foto.jpg|, e \verb|largura|, cujo
valor é \verb|100px|.

% ------------------------------------------------------------------------------
\section{Estrutura dos arquivos utilizados}
A estrutura dos arquivos utilizados pelo {\it Simddef}, assim como todos arquivo
XML tem início com uma declaração do tipo

\vspace{1.1\topsep}
\inputminted[fontsize = \footnotesize,
             bgcolor = cinzaclaro,
%             linenos = true,
             firstline = 1,
%             firstnumber = 1,
             lastline = 1]{xml}{codigos/falhas.sdd}

\noindent na qual a versão do XML utilizado e o tipo de codificação do arquivo
podem ser modificados a critério da aplicação. Após a {\it tag} de inicialização
do documento, a estrutura dos elementos dos arquivos varia de acordo com o
recurso a ser utilizado dentro do sistema.

O primeiro elemento do arquivo é composto pela {\it tag} \verb|Simddef|, a qual
possui dois atributos, denominados \verb|versao| e \verb|tipo|:

\vspace{1.1\topsep}
\inputminted[fontsize = \footnotesize,
             bgcolor = cinzaclaro,
%             linenos = true,
             firstline = 2,
%             firstnumber = 2,
             lastline = 2]{xml}{codigos/falhas.sdd}

O primeiro atributo desta {\it tag} poderá ser utilizado, por exemplo, em
futuras versões do sistema, para adequar a estrutura do arquivo que está sendo
lido às novas versões disponíveis. Já o segundo atributo é utilizado para
definir que tipo de arquivo está sendo carregado no sistema.

Nesta primeira versão, existem somente dois valores possíveis para o atributo
\verb|tipo|, são eles: \verb|modulos| e \verb|falhas|. Os elementos seguintes a
serem lidos dependem diretamente do valor deste atributo. Assim, tendo conhecido
esse valor, a classe que manipula os dados que estão sendo lidos poderá seguir
um dos dois caminhos distintos. O término da leitura do arquivo encerra quando a
{\it end-tag} \verb|</Simddef>| for lida.

% ------------------------------------------------------------------------------
\subsection{Arquivo de configuração de falhas}
O arquivo de configuração de falhas possui a seguinte estrutura:

\vspace{1.1\topsep}
\begin{minted}[fontsize = \footnotesize,
               bgcolor = cinzaclaro]{xml}
<?xml version="1.0" encoding="UTF-8"?>
<Simddef versao="1.0" tipo="falhas">
    <Falha>
        <local> ... </local>
        <abrv> ... </abrv>
        <descricao> ... </descricao>
    </Falha>
</Simddef>
\end{minted}
\vspace{1.1\topsep}

O elemento \verb|local| se refere ao local em que a falha poderá ocorrer, e pode
ter três valores possíveis: \verb|Sensor|, \verb|Atuador| ou \verb|Sistema|. Já
o elemento \verb|abrv| se refere a uma abreviatura estabelecida para aquela
falha, a qual será utilizada em determinadas localizações do sistema para evitar
que textos muito grandes apareçam na interface do usuário. Por fim, o elemento
\verb|descricao| se refere a uma descrição da falha. Este último elemento
poderá descrever alguns detalhes sobre como a falha ocorre e, possivelmente,
algum outro aspecto importante relacionado à falha.

O elemento \verb|Falha| poderá ser repetido tantas vezes quantas forem o número
de falhas cadastradas no sistema, desde que cada falha seja especificada a
partir dos três elementos citados anteriormente.

% ------------------------------------------------------------------------------
\subsection{Arquivo de configuração de módulos}
O arquivo de configuração dos módulos possui estrutura semelhante ao arquivo de
falhas. Contudo, sabe-se que poderão ser implementados diversos tipos de módulos
e cada um deles deverá estar associado à uma falha. Essa característica faz com
que possam existir {\it tags} diferentes no arquivo, dependendo do tipo de
módulo que está sendo carregado. Para um módulo neural, por exemplo, a estrutura
do arquivo poderia ser:

\vspace{1.1\topsep}
\begin{minted}[fontsize = \footnotesize,
               bgcolor = cinzaclaro]{xml}
<?xml version="1.0" encoding="UTF-8"?>
<Simddef versao="1.0" tipo="modulos">
    <Modulo>
      <tipo nome="RNA">
          <ordem>...</ordem>
      </tipo>
      <falha>...</falha>
      <arquivos qtde="3">
          <arq end="..."/>
          <arq end="..."/>
          <arq end="..."/>
      </arquivos>
   </Modulo>
</Simddef>
\end{minted}
\vspace{1.1\topsep}

Nesse caso, o elemento \verb|tipo| se refere ao tipo de módulo que está sendo
configurado, podendo ter como valor, por exemplo, \verb|RNA|, \verb|Fuzzy|,
\verb|Estatístico|, ou ainda qualquer outro nome dentre os módulos que venham a
ser implementados. O elemento \verb|ordem|, se refere a ordem da rede neural, a
qual determina o número de regressores que serão utilizados nas entradas e
saídas da rede. O elemento \verb|falha| associa o módulo à uma falha previamente
carregada a partir de sua abreviatura. O elemento \verb|arquivos| possui o
atributo \verb|qtde|, que indica o número de arquivos que serão carregados para
a configuração do módulo. No caso de uma rede neural, os arquivos poderão, por
exemplo, conter os pesos sinápticos e as funções de ativação de cada uma das
camadas. As {\it tags} \verb|arq|, com seu atributo \verb|end|, indicam o
endereço do arquivo que será carregado. O valor do atributo \verb|qtde| deverá
ser condizente com o número de {\it tags} \verb|arq| do módulo.

Assim como no arquivo de falhas, o elemento \verb|Modulo| pode se repetir tantas
vezes quantos forem o número de módulos cadastrados no sistema. Dependendo do
tipo de módulo, outras {\it tags} poderão ser adicionadas. A manipulação de cada
uma dessas {\it tags} poderá ser implementada na classe de manipulação dos
arquivos XML.

É importante ressaltar também que cada módulo deverá ser implementado pelo
usuário como uma classe que herda métodos e atributos de uma classe abstrata,
de acordo com a interface abaixo. Os métodos virtuais puros são os únicos
utilizados pelo sistema para processar as informações e exibir as falhas
detectadas ao usuário.

\vspace{1.1\topsep}
\inputminted[fontsize = \footnotesize,
             bgcolor = cinzaclaro,
             linenos = true,
             firstline = 1,
             firstnumber = 1,
             lastline = 49]{cpp}{codigos/modulo.h}

\inputminted[fontsize = \footnotesize,
             bgcolor = cinzaclaro,
             linenos = true,
             firstline = 50,
             firstnumber = 50,
             lastline = 69]{cpp}{codigos/modulo.h}
\vspace{1.1\topsep}

Dentre os métodos a serem implementados, o método \verb|curvas_a_exibir| não
possui nenhuma entrada e produz como saída uma {\it hash} que mapeia o número da
curva a ser exibida com seu respectivo nome. Esta função tem como principal
objetivo fornecer ao sistema uma maneira simples e direta de inserir os
valores desejados no gráfico e associar as curvas inseridas à legenda.

O método \verb|nome_tipo| também não possui entrada e fornece como saída o nome
do tipo do módulo que foi carregado. Este nome é utilizado na janela de
configuração dos módulos e na janela principal do sistema.

O método \verb|ler_arquivos| assume que todos os arquivos foram configurados
durante a criação do módulo, na leitura do arquivo XML, de tal maneira que
nenhuma entrada é necessária. Este método serve para fazer a leitura dos
arquivos e configurar o módulo de acordo com o seu tipo. No caso dos módulos
neurais, por exemplo, é este método quem cria a rede, atribui os pesos
sinápticos e configura as funções de ativação de cada camada. Como o intuito do
método é de apenas ler os arquivos e efetuar a configuração dos módulos, este
método não produz nenhuma saída.

Por fim, o método \verb|processar_saida| é quem efetivamente realiza o
processamento de todos os dados que foram carregados. Este método não é chamado
a partir do sistema, devendo ser utilizado dentro de alguma rotina interna da
classe que está sendo implementada como, por exemplo, ao final do método
\verb|ler_arquivos|.

% ------------------------------------------------------------------------------
\section{Exemplos}
Nas seções seguintes poderão ser visualizados exemplos completos dos arquivos de
configuração de falhas e módulos. As quebras de linha que existem em alguns
dos elementos \verb|descricao| foram ali colocadas apenas para evitar que a
formatação do código neste documento excedesse as margens de impressão.

% ------------------------------------------------------------------------------
\subsection{Arquivo de configuração de falhas}
\inputminted[fontsize = \footnotesize,
             bgcolor = cinzaclaro,
             linenos = true,
             samepage = false,
             firstline = 1,
             firstnumber = 1,
             lastline = 48]{xml}{codigos/falhas.sdd}

\inputminted[fontsize = \footnotesize,
             bgcolor = cinzaclaro,
             linenos = true,
             samepage = false,
             firstline = 49,
             firstnumber = 49,
             lastline = 72]{xml}{codigos/falhas.sdd}

% ------------------------------------------------------------------------------
\subsection{Arquivo de configuração de módulos}
\inputminted[fontsize = \footnotesize,
             bgcolor = cinzaclaro,
             linenos = true,
             samepage = false,
             firstline = 1,
             firstnumber = 1,
             lastline = 24]{xml}{codigos/modulos.sdd}

\inputminted[fontsize = \footnotesize,
             bgcolor = cinzaclaro,
             linenos = true,
             samepage = false,
             firstline = 25,
             firstnumber = 25,
             lastline = 79]{xml}{codigos/modulos.sdd}

\inputminted[fontsize = \footnotesize,
             bgcolor = cinzaclaro,
             linenos = true,
             samepage = false,
             firstline = 80,
             firstnumber = 80,
             lastline = 134]{xml}{codigos/modulos.sdd}

\inputminted[fontsize = \footnotesize,
             bgcolor = cinzaclaro,
             linenos = true,
             samepage = false,
             firstline = 135,
             firstnumber = 135,
             lastline = 146]{xml}{codigos/modulos.sdd}
