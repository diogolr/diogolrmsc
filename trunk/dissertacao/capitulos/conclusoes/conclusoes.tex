\mychapter{Conclusões}
\label{cap:conclusoes}

% ------------------------------------------------------------------------------
\section{Conclusões}
O presente trabalho foi desenvolvido com o intuito de fornecer um sistema DDF
para um sistema de tanques acoplados. Para isso, o sistema fez uso de uma
estrutura neural que, a partir dos valores disponíveis, indicava ao usuário os
momentos em que as falhas estavam ocorrendo.

O sistema proposto foi constituído por um módulo de identificação, o qual
realizava a inferência dos níveis dos tanques, e um conjunto de redes
especialistas, capazes de identificar cada uma das treze falhas listadas.

Dentre essas falhas, oito foram identificadas com facilidade e outras três
tiveram um desempenho satisfatório, com um pequeno problema de detecção que pode
ser facilmente resolvido através de {\it flags} binárias. As outras duas falhas
não foram identificadas corretamente, mas, em ambos os casos, o sistema consegue
detectar a falha corretamente para $T_1$.

Observa-se ainda que os resultados obtidos podem melhorar ainda mais quando
forem utilizados valores reais, oscilando dentro da faixa de valores em que a
rede foi treinada. Tal situação pode fazer com que não mais ocorra o problema de
detecção das FSeDG, FSeDO e FSiVzT, evitando portanto a utilização de {\it
flags} para contorná-lo.

Assim, pode-se dizer que o sistema teve um desempenho satisfatório, conseguindo
identificar cerca de 85\% das falhas propostas. Comprova-se, portanto, que as
redes neurais de múltiplas camadas são estruturas eficientes tanto para a
identificação de modelos quanto para a detecção e o diagnóstico de falhas.

Em contraponto, vale salientar que não foram realizados testes que avaliassem o
desempenho das redes de detecção atuando de maneira simultânea ou paralelamente.
Essa situação pode vir a gerar falsos positivos uma vez que as variáveis de
entrada das redes são idênticas. Outro aspecto que reforça essa teoria é que a
rede da primeira proposta de detecção não convergiu. Conforme dito
anteriormente, a não convergência das redes dessa proposta se deu,
provavelmente, em consequência da ausência de dados representativos no universo
das variáveis disponíveis, obtidas direta ou indiretamente a partir do processo.

% ------------------------------------------------------------------------------
\section{Perspectivas}
Após obter os resultados, com as falhas agindo de maneira individual sobre cada
uma das redes especialistas, o passo seguinte a ser dado envolve a aplicação de
sinais de excitação em todas as redes especialistas simultaneamente. Dessa
forma, a partir das saídas ativas de cada uma das redes, será possível
identificar as falhas que possuem um comportamento semelhante. 
 
Uma outra comparação pode ser realizada a partir da modificação das estruturas
neurais, utilizando novos tipos de rede, tais como as redes de função de base
radial, máquinas de vetor de suporte ou quaisquer outros modelos neurais que
possam ser adaptados ao problema de detecção e diagnóstico de falhas.

Tendo sido selecionada a melhor estrutura de detecção, o ambiente de simulação
desenvolvido em C++ poderá ser incrementado, adicionando a possibilidade da
simulação das falhas em tempo real. Para isso, faz-se necessário que as
estruturas neurais sejam adicionadas ao sistema e que sejam estabelecidas novas
formas de entrada dos parâmetros a serem modificados. Além disso, o sistema
poderá gerar sinais de saída de tal forma que as detecções das falhas sejam
observadas a partir de um sistema de monitoramento e supervisão externo. 

Com relação ao {\it Simddef}, apesar de os testes terem sido realizados apenas
com estruturas neurais, nada impede que novas técnicas venham a ser
implementadas, tais como técnicas {\it Fuzzy}, métodos estatísticos, novos tipos
de estruturas neurais, métodos de detecção por verificação de limites, técnicas
que utilizam observadores de estado, dentre outras. Além disso, a utilização de
arquivos de configuração com sintaxe XML facilita a integração desses novos
módulos ao sistema. Todo o código desenvolvido está disponível para {\it
download} em {\tt http://code.google.com/p/proeng-ufrn} para todos os
participantes do projeto Pró-Engenharias.

Uma vez que todo o sistema tenha sido testado com vários sinais de excitação e
diferentes tipos de módulos de detecção, pode-se agregar suas características à
um SCTF. Nesse caso os sinais gerados pelo sistema de DDF servirão como
``alarmes''. O SCTF, por sua vez, poderá realizar a reconfiguração dos
controladores, modificando os valores dos ganhos ou até mesmo suas estruturas,
de tal forma que o sistema continue funcionando de maneira correta até que a
falha tenha sido corrigida.
