\mychapter{Conclusões}
\label{cap:conclusoes}

% ------------------------------------------------------------------------------
\section{Conclusões}
%TODO dissertacao
% Fazer uma análise dos resultados obtidos com as redes atuando em paralelo e
% comparar o desempenho com a proposta 1 de detecção

O presente trabalho foi desenvolvido com o intuito de fornecer um sistema DDF
para um sistema de tanques acoplados. Para isso, o sistema fez uso de uma
estrutura neural que, a partir dos valores disponíveis, indicava ao usuário os
momentos em que as falhas estavam ocorrendo.

O sistema proposto foi constituído por um módulo de identificação, o qual
realizava a inferência dos níveis dos tanques, e um conjunto de redes
especialistas, capazes de identificar cada uma das treze falhas listadas.

Dentre essas falhas, oito foram identificadas com facilidade e outras três
tiveram um desempenho satisfatório, com um pequeno problema de detecção que pode
ser facilmente resolvido através de {\it flags} binárias. As outras duas falhas
não foram identificadas corretamente, mas, em ambos os casos, o sistema consegue
detectar a falha corretamente para $T_1$.

Observa-se ainda que os resultados obtidos podem melhorar ainda mais quando
forem utilizados valores reais, oscilando dentro da faixa de valores em que a
rede foi treinada. Tal situação pode fazer com que não mais ocorra o problema de
detecção das FSeDG, FSeDO e FSiVzT, evitando portanto a utilização de {\it
flags} para contorná-lo.

Assim, pode-se dizer que o sistema teve um desempenho satisfatório, conseguindo
identificar cerca de 85\% das falhas propostas. Comprova-se, portanto, que as
redes neurais de múltiplas camadas são estruturas eficientes tanto para a
identificação de modelos quanto para a detecção e o diagnóstico de falhas.

% ------------------------------------------------------------------------------
\section{Perspectivas}
% Modificação das estruturas neurais: Base Radial, SVM ...

Após obter os resultados parciais, com as falhas agindo de maneira individual
sobre cada uma das redes especialistas, o passo seguinte a ser dado envolve a
aplicação de sinais de excitação em todas as redes especialistas
simultaneamente. Dessa forma, a partir das saídas ativas de cada uma das redes,
será possível identificar as falhas que possuem um comportamento semelhante.
As falhas que possuírem essa característica podem ser classificadas a partir de
diferentes técnicas de detecção, utilizando tipos de módulos diferentes.

Tendo sido selecionada a melhor estrutura de detecção, o ambiente de simulação
desenvolvido em C++ poderá ser incrementado, adicionando a possibilidade da
simulação das falhas em tempo real. Para isso, faz-se necessário que as
estruturas neurais sejam adicionadas ao sistema e que sejam estabelecidas novas
formas de entrada dos parâmetros a serem modificados. Além disso, o sistema
poderá gerar sinais de saída de tal forma que as detecções das falhas sejam
observadas a partir de um sistema de monitoramento e supervisão externo. 

Com relação ao {\it Simddef}, apesar de os testes terem sido realizados apenas
com estruturas neurais, nada impede que novas técnicas venham a ser
implementadas, tais como técnicas {\it Fuzzy}, métodos estatísticos, novos tipos
de estruturas neurais, métodos de detecção por verificação de limites, técnicas
que utilizam observadores de estado, dentre outras. Além disso, a utilização de
arquivos de configuração com sintaxe XML facilita a integração desses novos
módulos ao sistema. Todo o código desenvolvido está disponível para {\it
download} em {\tt http://code.google.com/p/proeng-ufrn} para todos os
participantes do projeto Pró-Engenharias.

Uma vez que todo o sistema tenha sido testado com vários sinais de excitação e
diferentes tipos de módulos de detecção, pode-se agregar suas características à
um SCTF. Nesse caso os sinais gerados pelo sistema de DDF servirão como
``alarmes''. O SCTF, por sua vez, poderá realizar a reconfiguração dos
controladores, modificando os valores dos ganhos ou até mesmo suas estruturas,
de tal forma que o sistema continue funcionando de maneira correta até que a
falha tenha sido corrigida.
