% Comandos gerais --------------------------------------------------------------
\newcommand{\titulo}{Utilização de redes neurais artificiais para detecção e
diagnóstico de falhas}
\newcommand{\autor}{Diogo Leite Rebouças}
\newcommand{\emailautor}{\tt diogolr@dca.ufrn.br}
\newcommand{\orientador}{Fábio Meneghetti Ugulino de Araújo}
%\newcommand{\coorientador}{André Laurindo Maitelli}
\newcommand{\diaaprovacao}{21}
% Mês com inicial maiúscula (capa, folha de rosto)
\newcommand{\Mesaprovacao}{Janeiro}
% Mês com inicial minúscula (folha de assinaturas)
\newcommand{\mesaprovacao}{janeiro}
\newcommand{\anoaprovacao}{2011}
\let\Anoaprovacao\anoaprovacao
\newcommand{\dataaprovacao}{\diaaprovacao\ de \mesaprovacao\ de \anoaprovacao}
% Tipo de documento
\newcommand{\documento}{Defesa de Dissertação}
% Nome da ufrn
\newcommand{\ufrn}{Universidade Federal do Rio Grande do Norte}
% Nome do programa de pós-graduação
\newcommand{\ppgeec}{Programa de Pós-Graduação em Engenharia Elétrica e de
                     Computação}

% Configuração da fonte
%\renewcommand{\familydefault}{\sfdefault}

% Comandos matemáticos ---------------------------------------------------------
% Implicação em fórmulas
\newcommand{\implica}{$\quad\Rightarrow\quad$} %Meio de linha
\newcommand{\implicafim}{$\quad\Rightarrow$}   %Fim de linha
\newcommand{\tende}{$\rightarrow$}

% Fração com parenteses
\newcommand{\pfrac}[2]{\parent{\frac{#1}{#2}}}

% Transformada de Laplace e transformada Z
\newcommand{\lapl}{\pounds}
\newcommand{\transfz}{\mathcal{Z}}

% Sequências
\newcommand{\sequencia}[4]{$#1_{#2}$, $#1_{#3}$, \ldots, $#1_{#4}$}

% Outros ----------------------------------------------------------------------
\newcommand{\chave}[1]{\left\{#1\right\}}
\newcommand{\colchete}[1]{\left[#1\right]}
\newcommand{\parent}[1]{\left(#1\right)}

\newcommand{\rhoagua}{\rho_{\tiny \text{\tiny H}_2\text{\tiny O}}}
\newcommand{\reg}{\textsuperscript{\textregistered}}

\let\D\displaystyle

\newtheorem{teorema}{Teorema}
\newtheorem{exemplo}{Exemplo}
\newtheorem{definicao}{Definição}
\newtheorem{lema}{Lema}
\newcommand{\defin}[1]{\begin{definicao}#1\end{definicao}}

% Imagens exportadas pelo gnuplot ----------------------------------------------
\graphicspath{{imgs/resultados/eps/}}

% Hyperref ---------------------------------------------------------------------
\hypersetup{
    bookmarks=true,         % show bookmarks bar?
    unicode=false,          % non-Latin characters in Acrobat’s bookmarks
    pdftoolbar=true,        % show Acrobat’s toolbar?
    pdfmenubar=true,        % show Acrobat’s menu?
    pdffitwindow=false,     % window fit to page when opened
    pdfstartview={FitH},    % fits the width of the page to the window
    pdftitle={\titulo},     % title
    pdfauthor={\autor},     % author
    pdfsubject={mestrado},  % subject of the document
    pdfcreator={\autor},    % creator of the document
    pdfproducer={\autor},   % producer of the document
    pdfkeywords={Controle}, % list of keywords
    pdfnewwindow=true,      % links in new window
    colorlinks=false,       % false: boxed links; true: colored links
    linkcolor=black,        % color of internal links
    citecolor=black,        % color of links to bibliography
    filecolor=black,        % color of file links
    urlcolor=black          % color of external links
}

% Outros -----------------------------------------------------------------------
% Transparência dos tópicos
\setbeamercovered{transparent}

% Sumário entre as seções
\AtBeginSection[]
{
\begin{frame}
\frametitle{Sumário}
    \footnotesize
    \begin{columns}
        \column{0.45\textwidth}
            \tableofcontents[sections=1,currentsection]
            \vspace{0.25cm}
            \tableofcontents[sections=2,currentsection]
            \vspace{0.25cm}
            \tableofcontents[sections=3,currentsection]
        \column{0.45\textwidth}
            \tableofcontents[sections=4,currentsection]
            \vspace{0.25cm}
            \tableofcontents[sections=5,currentsection]
            \vspace{0.25cm}
            \tableofcontents[sections=6,currentsection]
            \vspace{0.25cm}
            \tableofcontents[sections=7,currentsection]
    \end{columns}
\end{frame}
}

% Sumário entre as subseções
%\AtBeginSubsection[]
%{
%\begin{frame}
%\frametitle{Sumário}
%\tableofcontents[currentsection,currentsubsection]
%\end{frame}
%}
