\newpage

\begin{center}

% Aqui não se usou \vfill porque o \vfill é construído internamente com
% o comando \vspace. Espaços verticais no início da folha com \vspace
% são ignorados. Para que isto não ocorra deve-se usar o \vspace*
% \vspace*{\fill} é como se fosse um \vfill*
\vspace*{\fill}

Divisão de Serviços Técnicos\\[1ex]
Catalogação da publicação na fonte.
UFRN / Biblioteca Central Zila Mamede

\vspace{2ex}

\begin{tabular}{|p{0.9\linewidth}|} \hline
\\
Rebouças, Diogo Leite.\\
\\
\hspace{1em} \titulo\ /\ Diogo Leite Rebouças -- Natal, RN, 2011. \\
\hspace{1em} 93 p. \\
\\
\hspace{1em} Orientador: \orientador \\
%\hspace{1em} Co-orientador: \coorientador \\
\\
\hspace{1em} Dissertação de mestrado - Universidade Federal do Rio Grande do
Norte. Centro de Tecnologia. Programa de Pós-Graduação em Engenharia Elétrica e
de Computação.
\\
\\
\hspace{1em} 1. Sistemas Críticos 2. Detecção de Falhas 3. Diagnóstico de Falhas
4. Redes Neurais Artificiais
\\
\\
RN/UF/BCZM \hfill CDU 004.932(043.2) \\ \hline
\end{tabular} 

\end{center}
