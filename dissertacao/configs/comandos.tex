% Comandos gerais --------------------------------------------------------------
\newcommand{\titulo}{Utilização de redes neurais artificiais para detecção e
diagnóstico de falhas}
\newcommand{\autor}{Diogo Leite Rebouças}
\newcommand{\emailautor}{\tt diogolr@dca.ufrn.br}
\newcommand{\orientador}{Fábio Meneghetti Ugulino de Araújo}
%\newcommand{\coorientador}{André Laurindo Maitelli}
\newcommand{\diaaprovacao}{30}
% Mês com inicial maiúscula (capa, folha de rosto)
\newcommand{\Mesaprovacao}{Junho}
% Mês com inicial minúscula (folha de assinaturas)
\newcommand{\mesaprovacao}{junho}
\newcommand{\anoaprovacao}{2011}
\let\Anoaprovacao\anoaprovacao
\newcommand{\dataaprovacao}{\diaaprovacao\ de \mesaprovacao\ de \anoaprovacao}
% Tipo de documento
\newcommand{\documento}{Dissertação de Mestrado}

% Configuração da fonte
%\renewcommand{\familydefault}{\sfdefault}

% Margens ----------------------------------------------------------------------
\setlength{\oddsidemargin}{3.5cm}
\setlength{\evensidemargin}{2.5cm}
\setlength{\textwidth}{15cm}
\addtolength{\oddsidemargin}{-1in}
\addtolength{\evensidemargin}{-1in}

\setlength{\topmargin}{2.0cm}
\setlength{\headheight}{1.0cm}
\setlength{\headsep}{1.0cm}
\setlength{\textheight}{22.7cm}
\setlength{\footskip}{1.0cm}
\addtolength{\topmargin}{-1in}

% Glossário --------------------------------------------------------------------
\makeglossary

% Capítulos --------------------------------------------------------------------
% Não aparecer o número na primeira página dos capítulos
\newcommand{\mychapter}[1]{\chapter{#1}\thispagestyle{empty}}
\newcommand{\mychapterstar}[1]{\chapter*{#1}\thispagestyle{empty}}

% Comandos matemáticos ---------------------------------------------------------
% Implicação em fórmulas
\newcommand{\implica}{\quad\Rightarrow\quad} %Meio de linha
\newcommand{\implicafim}{\quad\Rightarrow}   %Fim de linha
\newcommand{\tende}{\rightarrow}

% Fração com parenteses
\newcommand{\pfrac}[2]{\parent{\frac{#1}{#2}}}

% Transformada de Laplace e transformada Z
\newcommand{\lapl}{\pounds}
\newcommand{\transfz}{\mathcal{Z}}

% Sequências
\newcommand{\sequencia}[4]{$#1_{#2}$, $#1_{#3}$, \ldots, $#1_{#4}$}

% Comandos para o highlight dos códigos utilizando o pacote minted
\definecolor{cinzaclaro}{rgb}{0.95,0.95,0.95}
\renewcommand{\theFancyVerbLine}
{
    \sffamily\textcolor[rgb]{0.5,0.5,0.5}{\scriptsize\arabic{FancyVerbLine}}
}

% Outros ----------------------------------------------------------------------
\newcommand{\chave}[1]{\left\{#1\right\}}
\newcommand{\colchete}[1]{\left[#1\right]}
\newcommand{\parent}[1]{\left(#1\right)}

\newcommand{\rhoagua}{\rho_{\tiny \text{\tiny H}_2\text{\tiny O}}}

\let\D\displaystyle
\newcommand{\reg}{\textsuperscript{\textregistered}}

% Imagens exportadas pelo gnuplot ----------------------------------------------
\graphicspath{{imgs/resultados/eps/}}

% Hyperref ---------------------------------------------------------------------
\hypersetup{
    bookmarks=true,         % show bookmarks bar?
    unicode=false,          % non-Latin characters in Acrobat’s bookmarks
    pdftoolbar=true,        % show Acrobat’s toolbar?
    pdfmenubar=true,        % show Acrobat’s menu?
    pdffitwindow=false,     % window fit to page when opened
    pdfstartview={FitH},    % fits the width of the page to the window
    pdftitle={\titulo},     % title
    pdfauthor={Diogo Leite Reboucas},     % author
    pdfsubject={mestrado},  % subject of the document
    pdfcreator={Diogo Leite Reboucas},    % creator of the document
    pdfproducer={Diogo Leite Reboucas},   % producer of the document
    pdfkeywords={Controle}, % list of keywords
    pdfnewwindow=true,      % links in new window
    colorlinks=false,       % false: boxed links; true: colored links
    linkcolor=black,        % color of internal links
    citecolor=black,        % color of links to bibliography
    filecolor=black,        % color of file links
    urlcolor=black          % color of external links
}
