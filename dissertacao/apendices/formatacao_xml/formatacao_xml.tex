\chapter{Arquivos XML do \textbf{\textit{Simddef}}}
\label{ap:formatacao_xml}

% ------------------------------------------------------------------------------
\section{Arquivos XML}
XML ({\it eXtensible Markup Language}) é uma linguagem de marcação simples e
altamente flexível, criada pela W3C, derivada da {\it Standard Generalized
Markup Language} (SGML -- ISO 8879). Essa linguagem foi idealizada para gerar
linguagens de marcação para necessidades especiais e sua principal
característica é que as linguagens desconhecidas ou de pouco uso podem ser
facilmente definidas sem que haja necessidade de serem submetidas aos comitês de
padronização.

A formatação dos arquivos XML permite que os dados sejam organizados de forma
hierárquica, possibilitando que sejam criados arquivos para sua validação. Além
disso, o formato não depende de plataformas de {\it hardware} ou de {\it
software}, de tal forma que um banco de dados pode, através de uma aplicação,
escrever ou ler um arquivo desse tipo sem maiores complicações.

Por definição, um documento XML é composto por um conjunto de caracteres que,
quando dispostos de forma adequada, compõem uma árvore organizacional dos dados.
A estrutura gerada possui, dentre outros, conjuntos de {\it tags}, {\it
elementos} e {\it atributos}.

As {\it tags}, nada mais são do que as marcações que iniciam com \verb|<| e
terminam com \verb|>|, podendo ser dividida em {\it start-tags}, como por
exemplo \verb|<secao>|, {\it end-tags}, como por exemplo \verb|</secao>| e {\it
empty-element-tags}, como por exemplo \verb|<secao />|.

Os {\it elementos}, por sua vez, são as marcações dos componentes lógicos do
documento, que iniciam com uma {\it start-tag} e encerram com uma {\it end-tag},
podendo ainda serem compostos apenas por uma {\it empty-element-tag}. Os
caracteres entre as {\it tags}, quando existirem, constituem o conteúdo do
elemento e podem conter outros tipos de marcação, inclusive outros elementos,
dando origem aos {\it elementos filhos}. Um exemplo de elemento seria
\verb|<fala>Olá Mundo</fala>| ou apenas \verb|<quebra_linha />|

Já os {\it atributos} são marcações que consistem de um par {\it nome}/{\it
valor}, que existem no contexto das {\it start-tags} ou das {\it
empty-element-tags}. Como exemplo, no trecho 
\verb|<imagem end="foto.jpg" largura="100px" />|, existem dois atributos,
denominados \verb|end|, cujo valor é \verb|foto.jpg|, e \verb|largura|, cujo
valor é \verb|100px|.

% ------------------------------------------------------------------------------
\section{Estrutura dos arquivos utilizados}
A estrutura dos arquivos utilizados pelo {\it Simddef}, assim como todos arquivo
XML tem início com uma declaração do tipo

\vspace{\topsep}
\inputminted[fontsize = \footnotesize,
             bgcolor = cinzaclaro,
%             linenos = true,
             firstline = 1,
%             firstnumber = 1,
             lastline = 1]{xml}{codigos/falhas.sdd}

\noindent na qual a versão do XML utilizado e o tipo de codificação do arquivo
podem ser modificados a critério da aplicação. Após a {\it tag} de inicialização
do documento, a estrutura dos elementos dos arquivos varia de acordo com o
recurso a ser utilizado dentro do sistema.

O primeiro elemento do arquivo é composto pela {\it tag} \verb|Simddef|, a qual
possui dois atributos, denominados \verb|versao| e \verb|tipo|:

\vspace{\topsep}
\inputminted[fontsize = \footnotesize,
             bgcolor = cinzaclaro,
%             linenos = true,
             firstline = 2,
%             firstnumber = 2,
             lastline = 2]{xml}{codigos/falhas.sdd}

O primeiro atributo desta {\it tag} poderá ser utilizado, por exemplo, em
futuras versões do sistema, para adequar a estrutura do arquivo que está sendo
lido às novas versões disponíveis. Já o segundo atributo é utilizado para
definir que tipo de arquivo está sendo carregado no sistema.

Nessa primeira versão, existem dois valores possíveis para o atributo
\verb|tipo|, são eles: \verb|modulos| e \verb|falhas|. Os elementos seguintes a
serem lidos dependem diretamente do valor deste atributo. Assim, quando esse
valor é lido, a classe que manipula os dados que estão sendo lidos a partir do
arquivo XML poderá seguir um dos dois caminhos distintos. O término da leitura
do arquivo encerra quando a {\it end-tag} \verb|</Simddef>| for lida.

% ------------------------------------------------------------------------------
\subsection{Leitura do arquivo de configuração de falhas}

% ------------------------------------------------------------------------------
\subsection{Leitura do arquivo de configuração de módulos}

% ------------------------------------------------------------------------------
\section{Exemplos}
Nas seções seguintes poderão ser visualizados exemplos completos dos arquivos de
configuração de falhas e módulos.

% ------------------------------------------------------------------------------
\subsection{Arquivo de configuração de falhas}
\begin{listing}[H]
\inputminted[fontsize = \footnotesize,
             bgcolor = cinzaclaro,
             linenos = true,
             samepage = false ]{xml}{codigos/falhas.sdd}
\end{listing}

% ------------------------------------------------------------------------------
\subsection{Arquivo de configuração de módulos}
\begin{listing}[H]
\inputminted[fontsize = \footnotesize,
             bgcolor = cinzaclaro,
             linenos = true ]{xml}{codigos/modulos.sdd}
\end{listing}
