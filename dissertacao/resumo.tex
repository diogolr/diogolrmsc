\mychapterstar{Resumo}
Em um processo real, todos os recursos utilizados, sejam físicos ou
desenvolvidos em {\it software}, estão sujeitos a interrupções ou a
comprometimentos operacionais. Contudo, nas situações em que operam os sistemas
críticos, qualquer tipo de problema pode vir a trazer grandes consequências.
Sabendo disso, este trabalho se propõe a desenvolver um sistema capaz de
detectar a presença e indicar os tipos de falhas que venham a ocorrer em um
determinado processo. Para implementação e testes da metodologia proposta, um
sistema de tanques acoplados foi escolhido como modelo de estudo de caso. O
sistema a ser desenvolvido deverá gerar um conjunto de sinais que notifiquem o
operador do processo e que possam vir a ser pós-processados, possibilitando que
sejam feitas alterações nas estratégias ou nos parâmetros dos controladores. Em
virtude dos riscos envolvidos com relação à queima dos sensores, atuadores e
amplificadores existentes na planta real, o conjunto de dados das falhas foram
gerados computacionalmente e os resultados coletados a partir de simulações
numéricas do modelo do processo, não havendo risco de dano aos equipamentos. O
sistema será composto por estruturas que fazem uso de Redes Neurais Artificiais,
treinadas em modo {\it offline} pelo {\it software} matemático Matlab\reg.

\vspace{1.5ex}

\noindent {\bf Palavras-chave}: Sistemas Críticos, Detecção de Falhas, Diagnóstico de
Falhas, Redes Neurais Artificiais.
