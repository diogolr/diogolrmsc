% Comandos gerais --------------------------------------------------------------
\newcommand{\titulo}{Título do trabalho grande o suficiente para ter uma ideia
                     do que irá acontecer na capa}
\newcommand{\autor}{Diogo Leite Rebouças}
\newcommand{\orientador}{Fábio Meneghetti Ugulino de Araújo}
\newcommand{\coorientador}{André Laurindo Maitelli}
\newcommand{\diaaprovacao}{26}
% Mês com inicial maiúscula (capa, folha de rosto)
\newcommand{\Mesaprovacao}{Julho}
% Mês com inicial minúscula (folha de assinaturas)
\newcommand{\mesaprovacao}{julho}
\newcommand{\anoaprovacao}{2010}
\let\Anoaprovacao\anoaprovacao
\newcommand{\dataaprovacao}{\diaaprovacao\ de \mesaprovacao\ de \anoaprovacao}

% Configuração da fonte
%\renewcommand{\familydefault}{\sfdefault}

% Margens ----------------------------------------------------------------------
\setlength{\oddsidemargin}{3.5cm}
\setlength{\evensidemargin}{2.5cm}
\setlength{\textwidth}{15cm}
\addtolength{\oddsidemargin}{-1in}
\addtolength{\evensidemargin}{-1in}

\setlength{\topmargin}{2.0cm}
\setlength{\headheight}{1.0cm}
\setlength{\headsep}{1.0cm}
\setlength{\textheight}{22.7cm}
\setlength{\footskip}{1.0cm}
\addtolength{\topmargin}{-1in}

% Glossário --------------------------------------------------------------------
\makeglossary

% Capítulos --------------------------------------------------------------------
% Não aparecer o número na primeira página dos capítulos
\newcommand{\mychapter}[1]{\chapter{#1}\thispagestyle{empty}}
\newcommand{\mychapterstar}[1]{\chapter*{#1}\thispagestyle{empty}}

% Comandos matemáticos ---------------------------------------------------------
% Implicação em fórmulas
\newcommand{\implica}{\quad\Rightarrow\quad} %Meio de linha
\newcommand{\implicafim}{\quad\Rightarrow}   %Fim de linha
\newcommand{\tende}{\rightarrow}

% Fração com parenteses
\newcommand{\pfrac}[2]{\parent{\frac{#1}{#2}}}

% Transformada de Laplace e transformada Z
\newcommand{\lapl}{\pounds}
\newcommand{\transfz}{\mathcal{Z}}

% Sequências
\newcommand{\sequencia}[4]{$#1_{#2}$, $#1_{#3}$, \ldots, $#1_{#4}$}

% Outros ----------------------------------------------------------------------
\newcommand{\chave}[1]{\left\{#1\right\}}
\newcommand{\colchete}[1]{\left[#1\right]}
\newcommand{\parent}[1]{\left(#1\right)}
