\mychapterstar{Resumo}
Em um processo real, todos os recursos utilizados, sejam físicos ou
desenvolvidos em {\it software}, estão sujeitos a interrupções ou a
comprometimentos operacionais. Entretanto, nas situações em que operam os
sistemas críticos, qualquer tipo de problema pode vir a trazer grandes
consequências. Sabendo disso, este trabalho se propõe a desenvolver um sistema
capaz de detectar a presença e indicar a intensidade das falhas que venham a
ocorrer em um sistema de tanques acoplados, escolhido como modelo de estudo de
caso. O sistema a ser desenvolvido deverá gerar um conjunto de alarmes que
notifiquem o operador do processo e que possam vir a ser pós-processados,
possibilitando que sejam feitas alterações nas estratégias ou nos parâmetros de
controle. Em virtude dos riscos envolvidos com relação à queima dos sensores,
atuadores e amplificadores existentes na planta real, o conjunto de dados das
falhas serão gerados computacionalmente e os resultados serão coletados a partir
de simulações numéricas do modelo do processo. Ao final das simulações, será
realizada uma análise de desempenho do sistema considerando as falhas que
puderem ser implementadas no processo real sem que haja risco de dano aos
equipamentos. O sistema fará uso de Redes Neurais Artificiais para realizar a
detecção e o diagnóstico das falhas.

\vspace{1.5ex}

{\bf Palavras-chave}: Sistemas Críticos, Detecção e Diagnóstico de Falhas,
Sistema de Tanques, Redes Neurais Artificiais.
