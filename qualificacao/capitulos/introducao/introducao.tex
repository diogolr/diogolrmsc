\mychapter{Introdução}
\label{cap:introducao}

Segundo \citeasnoun{ribeiro:1999}, o termo automação está relacionado com a
substituição da mão-de-obra humana ou animal por uma máquina que realize uma
função equivalente. Partindo desse princípio, pode-se dizer que a automação
surge na sociedade em meados do século X, com os moinhos hidráulicos que
produziam farinha. Tal mecanismo, capaz de substituir o trabalho de dez a vinte
homens, levou a um crescimento da produção de alimentos nunca antes observado.

Desde então, o homem tem direcionado seu conhecimento para o desenvolvimento de
tecnologias que o auxiliem em suas atividades. Com a revolução industrial, a
partir da segunda metade do século XVIII, o processo de transformação e
desenvolvimento dessas tecnologias foi acelerado, de tal forma que o homem foi
capaz de produzir uma máquina a vapor para movimentar equipamentos industriais e
de fazer um martelo de 60 quilos dar 150 golpes por minuto \cite{goeking:2010}.

Por outro lado, a utilização de sistemas de controle remete a tempos ainda mais
antigos (300 a.C. a 250 a.C.), quando foram desenvolvidas as primeiras bóias
flutuadoras, o relógio de água de Ktsebios e uma lamparina a óleo que matinha o
nível de óleo combustível constante %\cite{ref1dorf, ref2dorf, ref3dorf}.

% ------------------------------------------------------------------------------
\section{Aspectos históricos}


\Glossary{$\alpha$}{Só para adicionar um símbolo}


% ------------------------------------------------------------------------------
\section{Importância da automação}

% ------------------------------------------------------------------------------
\section{Falhas}

% ------------------------------------------------------------------------------
\subsection{Tipos de falhas}

% ------------------------------------------------------------------------------
\subsection{Detecção}

% ------------------------------------------------------------------------------
\subsection{Métodos de detecção}

% ------------------------------------------------------------------------------
\subsection{Detecção de falhas com RNAs}
