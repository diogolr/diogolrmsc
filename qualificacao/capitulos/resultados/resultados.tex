\mychapter{Resultados parciais}
\label{cap:resultados}

% Dados para as simulações da qualificação
% Tempo de simulação de 105 segundos (totalizando 00:01:45)
% Valores utilizados => Observar folha


% Montar tabela com os valores dos parametros alterados para o treinamento e
% para a validação das falhas. Exemplo:
% 
% FAVK => km_treinamento = (0.7 a 1.1)*km_original (PRBS)
%         km_validacao = (0.75 a 1.05)*km_original (PRBS) => nao precisa
%                                                            especificar o valor
%                                                            exato, mas dizer
%                                                            que estava dentro
%                                                            da faixa do
%                                                            treinamento
%         km_resultado = (0.75)*km_original (Fixo)
%
% FSeDG => ganho_treinamento = (1 +- 0.2)*ganho_original (PRBS)
%          ganho_validacao = (1 +- 0.15)*ganho_original (PRBS)
%          ganho_resultado = 0.8*ganho_original (fixo)
%


% Mostrar que foram feitos 3 conjuntos de validacao e depois foram escolhidas as
% melhores redes => Mostrar tabelas com Falso positivo e falso negativo


% Quando for falar sobre o número de regressores escolhidos => Dizer que
% considere que os regressores estão representados internamente ao "sistema de
% identificação" e ao "sistema de DDF" na figura da composição do sistema (final
% do cap:sistema)

\begin{comment}
Neste capítulo será realizada uma análise comparativa dos resultados obtidos nas
duas propostas do sistema de DDF. Ao final, do capítulo a melhor estrutura será
escolhida para que seja feita uma análise mais detalhada dos resultados.
\end{comment}

Neste capítulo serão analisados os resultados parciais obtidos a partir da
implementação da segunda proposta do sistema de DDF. Para isso, em um primeiro
momento será mostrado como se deu a coleta dos dados de treinamento e validação
das redes especialistas e em seguida será feita uma comparação de desempenho. Ao
final do capítulo as melhores estruturas serão selecionadas para que se realize
uma análise um pouco mais detalhada acerca da detecção das falhas.

% ------------------------------------------------------------------------------
\section{Coleta dos dados}
Tanto para o processo de identificação quanto para o processo de detecção, o
primeiro passo a ser dado é a obtenção das amostras experimentais para o
treinamento supervisionado das redes neurais de inferência e de detecção.

Dessa maneira, realizou-se a coleta dos dados a partir da estimulação do sistema
simulado através da aplicação de sinais binários pseudo aleatórios ({Pseudo
Random Binary Signals} -- PRBS) à referência de cada um dos tanques e aos
parâmetros do sistema que simulam as falhas. Para a identificação, a faixa de
valores aplicados se deu do nível mínimo (zero) ao máximo (trinta). Já para a
detecção das falhas, os valores foram aplicados conforme Tab.
\ref{tab:valores_treinamento}. Nessa tabela os valores mínimos e máximos são
multiplicados pelo valor padrão e aplicados ao modelo. A última coluna mostra
qual será a representatividade da mudança dos valores.

\begin{table}[htb]
\caption{Valores aplicados para o treinamento das redes neurais de detecção.}
\label{tab:valores_treinamento}
\vspace{0.25cm}
\centering
\begin{threeparttable}
\begin{tabular}{|c|c|c|c|c|}
\hline
{\bf Falha} & {\bf Valor padrão} & {\bf Mínimo} & {\bf Máximo} & 
{\bf Representatividade}\\
\hline
\hline
FSeDG & 0,16\tnote{$*$} & 
0,8 & 1,2 & Até $\pm 6$ cm\\
\hline
FSeDO & 1,0 & -3,0 & 3,0 & Até $\pm 3$ cm\\
\hline
FSeSR & 1,0 & -0,03 & 0,03 & Até $\pm 9$ cm\\
\hline
FSeQ & 1,0 & 0,0 & 0,0 & -- \\
\hline
\hline
FADG & 1,0 & 0,8 & 1,0 & Até -3 Volts\\
\hline
FADO & 1,0 & -1,0 & 0,0 & Até -1 Volts\\
\hline
FASR & 1,0 & -0,03 & 0,03 & Até $\pm 0,45$ Volts\\
\hline
FAVK & $K_m$ & 0,7 & 1,1 & --\\
\hline
FAQ & 1,0 & 0,0 & 0,0 & --\\
\hline
\hline
FSiVzT & $a_{i_{\text{\tiny MED}}}$ & 
0,25 & 0,75 & 25 a 75\% de $a_{i_{\text{\tiny MED}}}$\\
\hline
FSiVrOS & $a_{i_{\text{\tiny MED}}}$ & 
0,75 & 1,25 & $\pm 25\%$ de $a_{i_{\text{\tiny MED}}}$\\
\hline
FSiVrGMP & 5,0\tnote{$*$} & 
0,8 & 1,0 & Até -3 Volts\\
\hline
FSiEOS & $a_{i_{\text{\tiny MED}}}$ & 
0,0 & 0,5 & --\\
\hline
\end{tabular}
\begin{tablenotes}
\item [$*$] Estabelecido pelo manual do fabricante.
\end{tablenotes}
\end{threeparttable}
\end{table}

Perceba que nas falhas dos atuadores, as tensões a serem aplicadas podem vir a
danificar a bomba. Por esse motivo não seria viável obter as amostras do
processo real, mas sim a partir de uma simulação.

Os sinais pseudo aleatórios gerados se mantiveram dentro dos limites
estabelecidos durante todo o intervalo de tempo da simulação. Para o processo de
identificação foram obtidas 6000 (seis mil) amostras, equivalentes à 10 (dez)
minutos de simulação. Já para a detecção o processo foi simulado durante 20
(vinte) minutos, o que correspondeu a obtenção de 12000 (doze mil) amostras.

De posse dos valores obtidos, iniciou-se a fase de treinamento das RNAs. Todas
as redes foram treinadas de modo {\it offline} com o {\it toolbox} de redes
neurais do {\it software} matemático Matlab\reg\ utilizando o algoritmo LMA. Ao
final de cada etapa de treinamento as redes eram submetidas à testes de
validação para que se fosse possível avaliar a capacidade de generalização.

% ------------------------------------------------------------------------------
\section{Análise das RNAs}
Para que se fossem estabelecidos critérios avaliativos significantes, ou seja,
para que existisse um número representativo, diversas redes neurais foram
treinadas com diferentes configurações considerando o número de neurônios na
camada oculta e a ordem do modelo. O número de redes treinadas corresponde
aquele especificado no final da Tab. \ref{tab:treinamentos}. 

\begin{table}[htb]
\centering
\caption[Número de redes neurais treinadas]{Número de redes neurais treinadas de
acordo com a ordem do modelo e o número de neurônios na camada oculta.}
\label{tab:treinamentos}
\vspace{0.25cm}
\begin{tabular}{|c|c|c|c|c|}
\hline
% Linha 1
\multirow{2}{*}{\bf Proposta} & 
\multirow{2}{*}{\bf Ordem} & 
{\bf Neurônios na} & 
{\bf Número de} & 
\multirow{2}{*}{\bf Total}\\
% Linha 2
& & {\bf camada oculta} & {\bf redes treinadas} &\\
\hline
\hline
\multicolumn{5}{|l|}{{\bf Identificação}}\\
\hline
\hline
\multirow{3}{*}{1} & 2 & 6/8/10 & \multirow{3}{*}{6} & \multirow{3}{*}{54}\\
\cline{2-3}
& 3 & 8/12/16 & &\\
\cline{2-3}
& 4 & 10/16/22 & &\\
\hline
\multirow{3}{*}{2} & 2 & 2/4/6 -- 6/8/10 & 
\multirow{3}{*}{6} & \multirow{3}{*}{108}\\
\cline{2-3}
& 3 & 4/6/8 -- 8/12/16 & & \\
\cline{2-3}
& 4 & 6/8/10 -- 10/16/22 & & \\
\hline
\multicolumn{5}{|l|}{{\bf Detecção}}\\
\hline
\hline
\multirow{3}{*}{2} & 2 & 8/12/16 & 
\multirow{3}{*}{6/falha} &
\multirow{3}{*}{702}\\
\cline{2-3}
& 3 & 14/18/22 & &\\
\cline{2-3}
& 4 & 20/24/28 & &\\
\hline
\hline
\multicolumn{4}{|r|}{{\bf Total}} & 864\\
\hline
\end{tabular}
\end{table}

Pode-se compreender essa tabela facilmente a partir de um exemplo: na primeira
proposta de identificação, considerando a linha de ordem igual 2, foram
treinadas 6 redes neurais cada vez que o número de neurônios na camada oculta
era alterado (6, 8 ou 10 neurônios). Assim, para cada ordem eram treinadas 18
redes neurais. Como foram testadas três ordens distintas, tem-se um total de 54
redes para a primeira proposta.

As únicas observações a serem feitas são que, na segunda proposta de
identificação existiam duas redes neurais a serem treinadas em cada etapa de
treinamento, conforme Fig. \ref{fig:ident_proposta_2}, e que a segunda proposta
de detecção considera-se um conjunto de 13 (treze) redes especialistas, sendo
uma para cada falha, conforme Fig. \ref{fig:detec_prop_2}. Em virtude desses
aspectos, o número de redes treinadas dobra da primeira para a segunda proposta
de identificação e é multiplicado por um fator de treze para a segunda proposta
de detecção.

% ------------------------------------------------------------------------------
\subsection{Identificação}

% ------------------------------------------------------------------------------
\subsection{Detecção de falhas}

% ------------------------------------------------------------------------------
\section{Melhores redes}

% ------------------------------------------------------------------------------
\section{Composição final}

% ------------------------------------------------------------------------------
\section{Detecções}

% ------------------------------------------------------------------------------
\section{Comparação das propostas}

% Falhas nos sensores ..........................................................
\begin{figure}[htb]
\footnotesize
\centering
\input{scripts/fsedg}
\vspace{1cm}
\caption{FSeDG (80\%)}
\label{fig:fsedg}
\end{figure}

\begin{figure}[htb]
\footnotesize
\centering
\input{scripts/fsedo}
\vspace{1cm}
\caption{FSeDO (-2 cm)}
\label{fig:fsedo}
\end{figure}

\begin{figure}[htb]
\footnotesize
\centering
\input{scripts/fsesr}
\vspace{1cm}
\caption{FSeSR ($\pm 2\%$)}
\label{fig:fsesr}
\end{figure}

\begin{figure}[htb]
\footnotesize
\centering
\input{scripts/fseq}
\vspace{1cm}
\caption{FSeQ (Ganho = 0)}
\label{fig:fseq}
\end{figure}

% Falhas nos atuadores .........................................................
\begin{figure}[htb]
\footnotesize
\centering
\input{scripts/fadg}
\vspace{1cm}
\caption{FADG (80\%)}
\label{fig:fadg}
\end{figure}

\begin{figure}[htb]
\footnotesize
\centering
\input{scripts/fado}
\vspace{1cm}
\caption{FADO (-0,5 Volts)}
\label{fig:fado}
\end{figure}

\begin{figure}[htb]
\footnotesize
\centering
\input{scripts/fasr}
\vspace{1cm}
\caption{FASR ($\pm 2\%$)}
\label{fig:fasr}
\end{figure}

\begin{figure}[htb]
\footnotesize
\centering
\input{scripts/favk}
\vspace{1cm}
\caption{FAVK (75\%)}
\label{fig:favk}
\end{figure}

\begin{figure}[htb]
\footnotesize
\centering
\input{scripts/faq}
\vspace{1cm}
\caption{FAQ (Ganho = 0)}
\label{fig:faq}
\end{figure}

% Falhas no sistema ............................................................
\begin{figure}[htb]
\footnotesize
\centering
\input{scripts/fsivzt}
\vspace{1cm}
\caption{FSiVzT ($a_{\tiny VZ} = \frac{a_{\tiny MED}}{2}$)}
\label{fig:fsivzt}
\end{figure}

\begin{figure}[htb]
\footnotesize
\centering
\input{scripts/fsivros}
\vspace{1cm}
\caption{FSiVrOS ($a' = \frac{a_{\tiny MED}}{2}$)}
\label{fig:fsivros}
\end{figure}

\begin{figure}[htb]
\footnotesize
\centering
\input{scripts/fsivrgmp}
\vspace{1cm}
\caption{FSiVrGMP (90\%)}
\label{fig:fsivrgmp}
\end{figure}

\begin{figure}[htb]
\footnotesize
\centering
\input{scripts/fsieos}
\vspace{1cm}
\caption{FSiEOS (25\%)}
\label{fig:fsieos}
\end{figure}
