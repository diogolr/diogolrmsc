\mychapter{Aspectos de projeto}
\label{cap:aspectos}

Com a demanda cada vez mais crescente com relação a eficiência, a qualidade dos
produtos e integração dos processos no setor industrial, aliada aos altos custos
envolvidos e as mais diversas necessidades de segurança, torna-se claro a
importância dos sistemas de supervisão e dos sistemas de detecção e diagnóstico
de falhas \cite{isermann:2006}.

Considerando tais aspectos, este capítulo mostrará os conceitos relacionados ao
desenvolvimento de um sistema de detecção e diagnóstico de falhas que fará uso
de Redes Neurais Artificiais. Inicialmente, serão mostradas as principais
terminologias utilizadas na área. Em seguida a atenção será voltada para as
Redes Neurais Artificiais e suas estruturas de identificação. Por fim, será
mostrado rapidamente como o sistema será estruturado para que sejam feitas as
simulações computacionais.

% ------------------------------------------------------------------------------
\section{Conceitos e terminologias}\label{sec:propriedades}
Segundo \citeasnoun{kaanich:2002}, os sistemas computacionais podem ser
caracterizados por cinco propriedades fundamentais: funcionalidade, usabilidade,
desempenho, custo e {\it dependabilidade}. Para \citeasnoun{kaanich:2002} {\it
apud} \citeasnoun{laprie:1992}, o termo {\it dependabilidade}, que nada mais é
do que a tradução literal do termo inglês {\it dependability}, está relacionado
com a capacidade de um sistema prestar um serviço que possa ser,
justificadamente, confiável. O serviço prestado se refere ao comportamento do
sistema percebido por seus usuários, os quais também serão sistemas (máquinas
físicas ou seres humanos), que interagem com o anterior.

De acordo com \citeasnoun{laprie:1994}, dependendo das aplicações envolvidas, o
termo dependabilidade pode ainda ser visto de acordo com suas diferentes, mas
complementares, propriedades:

\begin{itemize}
    \item {\bf Disponibilidade:} Prontidão para o uso;
    \item {\bf Confiabilidade:} Continuidade do serviço;
    \item \textbf{Proteção (\textit{safety}):} Não ocorrência de consequências
          catastróficas para o meio;
    \item {\bf Confidencialidade:} Não ocorrência da divulgação não-autorizada
          da informação;
    \item {\bf Integridade:} Não ocorrência de alterações indevidas da
          informação;
    \item {\bf Manutenção:} Aptidão para sofrer reparos e evolução.
\end{itemize}

A associação da integridade com a disponibilidade e a confidencialidade leva à
\textbf{Segurança (\textit{security})}.

Apesar do termo dependabilidade ser utilizado na maioria das vezes dessa forma,
o dicionário Michaelis, traduz tal termo como {\it confiabilidade} ou {\it
garantia de funcionamento}. Assim sendo, daqui para frente, utilizar-se-á a
palavra {\it confiável}, ou suas variações linguísticas, para se referir ao
termo {\it dependable}. Quando for desejado fazer referência a propriedade da
confiabilidade de um sistema, tal intenção será claramente especificada pelo
texto.

% ------------------------------------------------------------------------------
\subsection{Sistemática da dependabilidade}
\citeasnoun{avizienis:2000} e \citeasnoun{kaanich:2002} mostram que para se
desenvolver um Sistema Computacional Confiável (SCC, do inglês {\it Dependable
Computing System -- DCS}), faz-se uso de diferentes técnicas, tais como as
técnicas de: {\bf prevenção}, {\bf tolerância}, {\bf remoção} e {\bf previsão}
de falhas.

Além dessas técnicas, é válido também fazer referência aos termos {\bf avaria},
{\bf erro} e {\bf falha}, conceitualmente abordados de diferentes maneiras por
vários autores. Ao longo das Seções \ref{sec:avaria_erro_falha} e
\ref{sec:tecnicas}, os conceitos envolvidos relativos as técnicas e a essa
nomenclatura serão melhor explicados.

\begin{comment}
\begin{itemize}
    \item {\bf Técnicas de prevenção:} Como prevenir a ocorrência ou a introdução de
          falhas;
    \item {\bf Técnicas de tolerância:} Como oferecer um serviço ``correto'' na
          presença de falhas;
    \item {\bf Técnicas de remoção:} Como reduzir o número ou atenuar a gravidade
          das falhas;
    \item {\bf Técnicas de previsão:} Como estimar o número atual, a incidência
          futura e as consequências das prováveis falhas.
\end{itemize}
\end{comment}

Diante do que foi exposto, \citeasnoun{avizienis:2000} separa os termos acima
relacionados em três grupos: os atributos (propriedades), as ameaças e os meios
pelos quais a dependabilidade é atingida. Tal divisão pode ser observada na Fig.
\ref{fig:div_avizienis}.

\begin{figure}[htb]
\centering
\footnotesize
\[
\text{Dependabilidade}
\left\{
\begin{array}{l}
\text{Atributos}
    \left\{
    \begin{array}{l}
        \text{Disponibilidade}\\
        \text{Confiabilidade}\\
        \text{Proteção}\\
        \text{Confidencialidade}\\
        \text{Integridade}\\
        \text{Manutenção}\\
        \text{Segurança}
    \end{array}
    \right.
\\
\\
\text{Ameaças}
    \left\{
    \begin{array}{l}
        \text{Avaria}\\
        \text{Erro}\\
        \text{Falha}
    \end{array}
    \right.
\\
\\
\text{Meios}
    \left\{
    \begin{array}{l}
        \text{Prevenção}\\
        \text{Tolerância}\\
        \text{Remoção}
        \text{Previsão}
    \end{array}
    \right.
\end{array}
\right.
\]
\caption{Taxonomia da dependabilidade}
\label{fig:div_avizienis}
\end{figure}

O primeiro grupo, cujos elementos foram explicados na seção
\ref{sec:propriedades}, permite expressar as propriedades esperadas de um
sistema confiável e analisar suas qualidades. Já o segundo grupo traz os termos
utilizados para expressar características indesejadas -- mas em princípio não
inesperadas -- que causa ou fazem com que um sistema passe a ser não-confiável.
Por fim, o terceiro grupo exibe os meios ou as técnicas pelas quais torna-se
possível oferecer um serviço confiável.

% ------------------------------------------------------------------------------
\section{Avarias, erros e falhas}\label{sec:avaria_erro_falha}
Em um processo real, todos os recursos utilizados, sejam físicos ou
implementados em {\it software}, estão sujeitos a interrupções ou a
comprometimentos operacionais.

Em sistemas críticos, tais como as aeronaves ou as usinas nucleares, essas
situações fazem com que pequenos ``deslizes operacionais'' representem grandes
consequências catastróficas. Dentre os exemplos mais conhecidos, pode-se
destacar os acidentes do {\it Airbus 320} da TAM em 2007, que não conseguiu
parar ao aterrissar no aeroporto de Congonhas, São Paulo, no qual morreram 199
pessoas (12 em solo e 187 no avião) e o desastre da usina nuclear de Chernobil,
que liberou cerca de 400 vezes mais contaminação do que a bomba nuclear que foi
lançada em de Hiroshima.

Esses exemplos fazem-nos refletir sobre como os sistemas de controle podem
evoluir para evitar que catástrofes ainda maiores venham a ocorrer. Ou seja,
como as propriedades de um sistema de controle confiável poderão ser mantidas
mesmo na presença de avarias, erros e falhas no processo.

Segundo \citeasnoun{nelio:2002}, apesar do termo falha ser utilizado, em muitos
dos casos, como um termo vago, abrangendo também o significado de avarias e
erros, existe uma diferença entre esses conceitos que deve ser destacada.

O termo {\bf avaria} ({\it failure}) deve ser utilizado para indicar que houve
um desvio do comportamento no sistema, o que o torna incapaz de fornecer o
serviço para o qual foi designado. Um {\bf erro} ({\it error}), entretanto, está
relacionado com estado do sistema e pode levar a uma avaria. De maneira
resumida, se há um erro no estado do sistema, então existe uma sequência de
ações que podem ser executadas e que levarão a avarias, a não ser que medidas de
correção venham a ser tomadas. Por fim, mas não menos importante, o termo {\bf
falha} ({\it fault}) é a causa de um erro e está associado à noção de defeitos.
Normalmente, diz-se que falha pode ser definida como sendo um defeito que possui
o potencial de gerar erros \cite{nelio:2002,weber:2002}.

Alguns autores nacionais costumam traduzir os termos {\it failure} como falha e
{\it fault} como falta. Entretanto, costuma-se falar em sistemas de controle
tolerante a ``falhas'' e não em sistemas de controle tolerantes a ``faltas''.
Observa-se então a necessidade de um maior cuidado com relação a utilização
dessas palavras, pois {\it failures} não podem ser toleradas.

\begin{comment}
Por esse motivo, ao longo do texto, quando for necessário se referir ao termo
{\it failure}, será utilizada a palavra avaria. Já o termo falha será utilizado
de maneira mais abrangente, envolvendo os conceitos de erro e de falha
explicados anteriormente.
\end{comment}

Como em muitos dos casos os sistemas são compostos de subsistemas, é comum se
observar que uma falha leva a um erro que por sua vez pode levar a uma avaria,
que gera novas falhas e dá início a uma reação em cadeia, tal como a Fig.
\ref{fig:reacao_cadeia}. Contudo, nem sempre uma falha conduz a um erro, assim
como nem sempre um erro conduz a uma avaria, mas todos os erros resultam de
falhas e todas as avarias resultam de erros.

\begin{figure}[htb]
\centering
\[
\ldots
\quad\longrightarrow\quad
\text{Falha} 
\quad\longrightarrow\quad
\text{Erro}
\quad\longrightarrow\quad
\text{Avaria}
\quad\Longrightarrow\quad
\text{Falha}
\quad\longrightarrow\quad
\ldots
\]
    \caption{Reação em cadeia das falhas, erros e avarias.}
    \label{fig:reacao_cadeia}
\end{figure}

\citeasnoun{weber:2002} comenta que as falhas são inevitáveis, uma vez que os
componentes físicos do sistema envelhecem e estão sempre sujeitos as
interferências externas, ambientais e humanas. Para ela, assim como os sistemas
físicos, os {\it softwares} também são vítimas, pois estão a mercê da alta
complexidade dos processos e da fragilidade humana em trabalhar com grande
volume de detalhes de especificação/operação.

% ------------------------------------------------------------------------------
\subsection{Tipos de falhas}


% ------------------------------------------------------------------------------
\section{Prevenção, tolerância, remoção e previsão de falhas}
\label{sec:tecnicas}

% ------------------------------------------------------------------------------
\subsection{Detecção}

% ------------------------------------------------------------------------------
\subsection{Métodos de detecção}

% ------------------------------------------------------------------------------
\subsection{Detecção de falhas com RNAs}

% ------------------------------------------------------------------------------
\section{Aplicações de detecção de falhas com RNAs}

% ------------------------------------------------------------------------------
\section{Redes neurais artificiais}

% ------------------------------------------------------------------------------
\subsection{Tipos de RNAs para identificação}

% ------------------------------------------------------------------------------
\subsection{Tipos de RNAs para detecção}

% ------------------------------------------------------------------------------
\section{Simulações computacionais}

% ------------------------------------------------------------------------------
\subsection{Método de Runge-Kutta de 4\textordfeminine ordem}
