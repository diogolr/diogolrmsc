\mychapter{Aspectos de projeto}
\label{cap:aspectos}

Com a demanda cada vez mais crescente com relação a eficiência, a qualidade dos
produtos e integração dos processos no setor industrial, aliada aos altos custos
envolvidos e as mais diversas necessidades de segurança, torna-se claro a
importância dos sistemas de supervisão e dos sistemas de detecção e diagnóstico
de falhas \cite{isermann:2006}.

Considerando tais aspectos, este capítulo mostrará os conceitos relacionados ao
desenvolvimento de um sistema de detecção e diagnóstico de falhas que fará uso
de Redes Neurais Artificiais. Inicialmente, serão mostradas as principais
terminologias utilizadas na área. Em seguida a atenção será voltada para as
Redes Neurais Artificiais e suas estruturas de identificação. Por fim, será
mostrado rapidamente como o sistema será estruturado para que sejam feitas as
simulações computacionais.

% ------------------------------------------------------------------------------
\section{Conceitos e terminologias}
Segundo \citeasnoun{kaanich:2002}, os sistemas computacionais podem ser
caracterizados por cinco propriedades fundamentais: funcionalidade, usabilidade,
desempenho, custo e {\it dependabilidade}. Para \citeasnoun{kaanich:2002} {\it
apud} \citeasnoun{laprie:1992}, o termo {\it dependabilidade} está relacionado
com a capacidade de um sistema prestar um serviço que possa ser,
justificadamente, confiável. O serviço prestado se refere ao comportamento do
sistema percebido por seus usuários, os quais também serão sistemas
(máquinas físicas ou seres humanos), que interagem com o anterior.

De acordo com \citeasnoun{laprie:1994}, dependendo das aplicações envolvidas, o
termo dependabilidade pode ainda ser visto de acordo com suas diferentes, mas
complementares, propriedades:

\begin{itemize}
    \item {\bf Disponibilidade:} Prontidão para o uso;
    \item {\bf Confiabilidade:} Continuidade do serviço;
    \item \textbf{Proteção (\textit{safety}):} Não ocorrência de consequências
          catastróficas para o meio;
    \item {\bf Confidencialidade:} Não ocorrência da divulgação não-autorizada
          da informação;
    \item {\bf Integridade:} Não ocorrência de alterações indevidas da
          informação;
    \item {\bf Manutenção:} Aptidão para sofrer reparos e evolução.
\end{itemize}

A associação da integridade com a disponibilidade e a confidencialidade leva à
\textbf{Segurança (\textit{security})}.

% ------------------------------------------------------------------------------
\subsection{Taxonomia da dependabilidade}
O termo dependabilidade nada mais é do que a tradução literal do termo inglês
{\it dependability}. Entretanto, o dicionário Michaelis, traduz tal termo como
{\it confiabilidade} ou {\it garantia de funcionamento}. Assim sendo, daqui para
frente, utilizar-se-á a palavra {\it confiável}, ou suas variações linguísticas,
para se referir ao termo {\it dependable}. Quando for desejado fazer referência
a propriedade da confiabilidade de um sistema, tal intenção será claramente
especificada pelo texto.

\citeasnoun{avizienis:2000} e \citeasnoun{kaanich:2002} mostram que para se
desenvolver um Sistema Computacional Confiável (SCC, do inglês {\it Dependable
Computing System -- DCS}), faz-se uso de diferentes técnicas, tais como as
técnicas de: {\bf prevenção}, {\bf tolerância}, {\bf remoção} e {\bf previsão}
de falhas. Todas essas técnicas serão brevemente explicadas na Seção
\ref{sec:tecnicas}.

\begin{comment}
\begin{itemize}
    \item {\bf Técnicas de prevenção:} Como prevenir a ocorrência ou a introdução de
          falhas;
    \item {\bf Técnicas de tolerância:} Como oferecer um serviço ``correto'' na
          presença de falhas;
    \item {\bf Técnicas de remoção:} Como reduzir o número ou atenuar a gravidade
          das falhas;
    \item {\bf Técnicas de previsão:} Como estimar o número atual, a incidência
          futura e as consequências das prováveis falhas.
\end{itemize}
\end{comment}

Além das propriedades descritas na seção anterior e das técnicas acima citadas,
deve-se conhecer também os conceitos de {\bf defeito} (ou {\bf avaria}), {\bf
erro} e {\bf falha}, a serem explicados na Seção \ref{sec:avaria_erro_falha}.

\citeasnoun{avizienis:2000} agrupa os termos acima discutidos em três partes: os
atributos (propriedades), as ameaças e os meios pelos quais a dependabilidade é
atingida. Tal divisão pode ser observada na Fig. \ref{fig:div_avizienis}.

% ------------------------------------------------------------------------------
\subsection{Tipos de falhas}

% ------------------------------------------------------------------------------
\subsection{Detecção}

% ------------------------------------------------------------------------------
\subsection{Métodos de detecção}

% ------------------------------------------------------------------------------
\subsection{Detecção de falhas com RNAs}

% ------------------------------------------------------------------------------
\section{Aplicações de detecção de falhas com RNAs}

% ------------------------------------------------------------------------------
\section{Redes neurais artificiais}

% ------------------------------------------------------------------------------
\subsection{Tipos de RNAs para identificação}

% ------------------------------------------------------------------------------
\subsection{Tipos de RNAs para detecção}

% ------------------------------------------------------------------------------
\section{Simulações computacionais}

% ------------------------------------------------------------------------------
\subsection{Método de Runge-Kutta de 4\textordfeminine ordem}
